\section{Практика 6}
\subsection{Ряды Лорана}
Дано: $0 \leq R_1 < R_2 \leq \infty, \Omega = \{z: R_1 < |z - a| < R_2\}, a \in \CC$

\begin{exmpl}
    $R_1 = 1, R_2 = 2, a = 0 \leadsto$ диск  
\end{exmpl}

\begin{exmpl}
    $R_1 = 1, R_2 = \infty \leadsto \CC $ без круга 
\end{exmpl}

\begin{exmpl}
    $R_1 = 0, R_2 = \infty \leadsto \CC$ без точки 
\end{exmpl}

\begin{thm}
    $\exists c_k \in \CC, k \in \Z: f = \sum\limits_{k \in \Z}c_k(z - a)^k$, причем ряд сходится всюду в окрестности изолированной особой точки
\end{thm}

\begin{remrk}
    Сходиомсть понимается в следующем смысле: сходятся ряды $\sum\limits_{k \geq 0} c_k(z - a)^k, \sum\limits_{k < 0}c_k(z - a)^k$
\end{remrk}

\begin{df}
    Главная часть ряда Лорана --- это та, которая неограничена в кольце $\Omega$ (либо $\sum_{k \geq 0}$, либо $\sum_{k < 0}$)
\end{df}

\begin{exmpl}
    Найти главную часть ряда Лорана функции $f = \frac{(z + 1)^2}{\sin^2 z}$ в кольце $\pi > |z| > 0$ 
\end{exmpl}
\begin{sol}
    \begin{align*}
        f = \sum\limits_{k \in \Z}c_kz^k \\
        c_{-3} = 0, c_{-4} = 0, \dots \textrm{ --- примем как факт}\\
        f = \underbrace{\frac{c_{-2}}{z^2} + \frac{c_{-1}}{z}}_{\textrm{главная часть}} + c_0 + c_1z + \dots \\
        c_{-2} = \lim\limits_{z \to 0}\frac{z^2 \cdot (z + 1)^2}{\sin^2 z} = 1 \\
        g = f(z) - \frac{c_{-2}}{z^2} = f(z) - \frac{1}{z^2} \\ 
        c_{-1} = \lim\limits_{z \to 0}(z \cdot g(z)) = \lim\limits_{z \to 0} \left( \frac{(z + 1)^2}{\sin^2 z} - \frac{1}{z^2}\right)z = \\ 
        = \lim\limits_{z \to 0}\cfrac{\frac{z^2(z + 1)^2}{\sin^2 z} - 1}{z - 0} = \left[ \left( \frac{z^2(z + 1)^2}{\sin^2 z} \right)'_z (0) \right] = \\
        = \lim\limits_{z \to 0} \frac{z^2(z + 1)^2 - \sin^2 z}{z \cdot \sin^2 z} = \lim\limits_{z \to 0} \frac{z^2(z+1)^2 - \sin^2 z}{z^3} = \\
        = \lim\limits_{z \to 0} \frac{z^2(z + 1)^2 - z^2 _ O(z^4)}{z^3} = \lim\limits_{z \to 0} \frac{z^2(z^2 + 2z + 1) - z^2}{z^3}= \\ 
        = \lim\limits_{z \to 0} \frac{z^4 + 2z^3 + z^2 - z^2}{z^3} = 2
    \end{align*}

    $\then$ главная часть ряда Лорана: $\cfrac{1}{z^2} + \cfrac{2}{z}$
\end{sol}
\subsubsection{Классификация изолированных особых точек}
$a \in \CC, f$ --- аналитична в проколотой окрестности $a$

\begin{center}
    \begin{tabular}{| c | c | c | c |}
        \hline
        тип & описание в терминах $f$ & ряд Лорана & пример \\
        \hline
        устранимая особая точка & $f$ --- ограничена в окрестности $a$ & $f = \sum\limits_{0}^{\infty}c_k(z - a)^k$ & $\frac{\sin z}{z}, a = 0$ \\
        \hline
        полюс порядка $n$ & $f \sim \frac{c}{(z - a)^n}, z \to a$ & $f = \sum\limits_{-n}^{\infty}c_k(z - a)^k, c_{-n} \neq 0$ & $\frac{1}{\cos^n z}, a = \frac{\pi}{2}$\\
        \hline
        существенная особая точка & $\overline{\lim\limits_{z \to a}}\big|f(z)(z - a)^n\big| = +\infty \; \forall \; n$ & $f = \sum\limits_{-\infty}^{\infty}c_k(z - a)^k$ & $e^{\frac{1}{z}}, a = 0; e^z, a = \infty$ \\
        \hline
    \end{tabular}
\end{center}

\begin{remrk}
    Для существенной особой точки существует бесконечно много коэффициентов $c_{-n} \neq 0, n > 0$
\end{remrk}

\begin{exmpl}
    Разложить в ряд Лорана $f = \cfrac{z^4 + 1}{(z - 1)(z + 2)}$ в кольце $1 < |z| < 2$
\end{exmpl}

\begin{sol}
    Ищем разлолжение $f = \sum\limits_{k \in \Z}c_k(z - a)^k$
    \begin{gather*}
        f = (z^4 + 1)\left( -\frac{1}{z - 1} + \frac{1}{z + 2} \right) \left( -\frac{1}{3} \right) = (z^4 + 1)\left( -\frac{1}{z} \cdot \frac{1}{1 - \frac{1}{z}} + \frac{1}{2} \cdot \frac{1}{1 + \frac{z}{2}} \right) \left( -\frac{1}{3} \right) = \\
        = (z^4 + 1)\left( -\frac{1}{z}\sum\limits_{k = 0}^{\infty}\frac{1}{z^k} + \frac{1}{2}\sum\limits_{k = 0}^{\infty} (-\frac{z}{2})^k\right) (-\frac{1}{3}) = \\
        = (z^4 + 1)\sum\limits_{k \in \Z}b_k z^k = \left[ \sum\limits_{k \in \Z}c_k z^k ?\right] \\ 
        b_k = \begin{cases}
            \frac{1}{3}, k < 0 \\ 
            -\frac{1}{6}(-\frac{1}{2})^k, k \geq 0
        \end{cases} \\ 
        = \sum b_k z^{k + 4} + \sum b_k z^k = \sum\limits_{k \in \Z}(b_{k - 4} + b_k) z^k, c_k = b_{k - 4} + b_k
    \end{gather*}
\end{sol}

\begin{exmpl}
    Разложить в ряд Лорана $f = \frac{1}{z - b}$ в кольце $|z - a| > |b - a|$
\end{exmpl}
\begin{sol}
    \begin{gather*}
        f = \frac{1}{z - b} = \frac{1}{z - a + a - b} = \frac{1}{z - a} \cdot \frac{1}{1  + \frac{a - b}{z - a}} = \\ 
        = \frac{1}{z - a}\left( \sum\limits_{k = 0}^{\infty} \left(-\frac{a - b}{z - a}\right)^k \right) = \\
        = \sum\limits_{k = 0}^{\infty} \frac{(b - a)^k}{(z - a)^{k + 1}}
    \end{gather*}
\end{sol}

\begin{exmpl}
    Разложить в ряд Лорана $f = \frac{1}{(z - b)^2}$ в кольце $|z - a| > |b - a|$
\end{exmpl}
\begin{sol}
    Заметим, что мы можем продиффиренцировать ряд для $f = \frac{1}{z - b}$ и результат с противоположынм знаком --- это ряд для $f = \frac{1}{(z - b)^2}$

    \begin{gather*}
        - \left( \sum\limits_{k = 0}^{\infty} \frac{(b - a)^k}{(z - a)^{k + 1}} \right)' = (k + 1)\sum\limits_{k = 0}^{\infty} \frac{(b - a)^k}{(z - a)^{k + 2}} = \\ 
        = \sum\limits_{-\infty}^{-2}c_n(z - a)^n, c_n = (b - a)(n \pm 1) \textrm{надо точно посчитать плюс тут или минус}
    \end{gather*}
\end{sol}

\begin{exmpl}
    Для $f(z) = z^3 \cos \left( \frac{1}{z - 2} \right)$: доказать, что $z = 2$ --- это полюс, найти его кратность, вычислить главную часть ряда Лорана.
\end{exmpl}
\begin{sol}
    
\end{sol}